% TODO: 目前内容不足以凑满两页(正反 A4 纸),所以有些内容注释掉了,之后完善

\documentclass{article}
\setlength\parindent{0pt}
\pagenumbering{gobble}

\usepackage{geometry}
\geometry{papersize={210mm,297mm}}
\geometry{left=9mm,right=9mm,top=9mm,bottom=9mm}

\usepackage{fontspec}
\setmainfont{DejaVu Sans}[Scale=0.9]

\usepackage{xeCJK}
\CJKfamily{zhsong}

\usepackage[svgnames]{xcolor}
\usepackage{longfbox}
\usepackage{epstopdf}
\usepackage{float}
\usepackage{xpatch}
\usepackage{tipa}
\usepackage[export]{adjustbox}
\usepackage{enumitem}
\usepackage{mdframed}

\usepackage{multirow}
\usepackage{tabularx}
\renewcommand\tabularxcolumn[1]{m{#1}} % for vertical centering text in X column

\usepackage{multicol}
\usepackage[most]{tcolorbox}
\usepackage{amssymb}
\setlength{\columnsep}{3mm}

\usepackage[explicit,compact]{titlesec}
\titleformat{\section}{\normalfont\bfseries}{}{0pt}{【#1】}

% 啊!万能的 StackExchange!
% https://tex.stackexchange.com/questions/475466/latex-three-column-layout-merging-two-of-them-at-the-begining
%
\newlength{\abstractwidth}
\newlength{\columnshrink}
\newsavebox{\twocolinsert}
%
\makeatletter
\newlength{\resized@col}
\newcounter{column@count}
%
\xpatchcmd{\multi@column@out}{
	\process@cols\mult@gfirstbox{%
		\setbox\count@
		\vsplit\@cclv to\dimen@
		\set@keptmarks
		\setbox\count@
		\vbox to\dimen@
		{\unvbox\count@ \ifshr@nking\vfilmaxdepth\fi}%
	}%
}{
	\process@cols\mult@gfirstbox{%
		\global\advance\c@column@count\@ne
		\resized@col\dimen@%
		\ifnum\c@column@count=\tw@
				\advance\resized@col-\columnshrink
		\fi%
		\setbox\count@
		\vsplit\@cclv to\resized@col
		\set@keptmarks
		\setbox\count@
		\vbox to\dimen@{
			\ifnum
				\c@column@count=\tw@ \vspace*{\columnshrink}
			\fi
			\unvbox\count@
			\ifshr@nking\vfilmaxdepth\fi
		}%
	}%
}{\typeout{Success}}{\typeout{Failure}}
\makeatother

% for designing header
\newsavebox\mysavebox
\newenvironment{imgminipage}[2][]{%
   \def\imgcmd{\includegraphics[width=\wd\mysavebox, height=\dimexpr\ht\mysavebox+\dp\mysavebox\relax, #1]{#2}}%
   \begin{lrbox}{\mysavebox}%
   \begin{minipage}%
}{%
   \end{minipage}
   \end{lrbox}%
   \sbox\mysavebox{\setlength{\fboxrule}{0pt}\fbox{\usebox\mysavebox}}%
   \mbox{\rlap{\raisebox{-\dp\mysavebox}{\imgcmd}}\usebox\mysavebox}%
}

\renewcommand{\labelitemi}{$\blacktriangleright$}

\tcbset{
    frame code={}
    center title,
    left=0pt,
    right=0pt,
    top=6pt,
    bottom=0pt,
    colback=gray!40,
    colframe=white,
    enlarge left by=0mm,
    boxsep=0pt,
    arc=0pt,outer arc=0pt,
}

\begin{document}
\begin{multicols*}{3}

	\setlength{\abstractwidth}{2\linewidth}
	\addtolength{\abstractwidth}{\columnsep}
	\savebox{\twocolinsert}{
	\begin{minipage}{\abstractwidth}
		\noindent 核准日期:2025 年 05 月 03 日
		\newline 修改日期:2025 年 05 月 03 日
		\newline

		 %\begin{mdframed}[leftline=false, rightline=false, innertopmargin=0pt, innerbottommargin=0pt, innerrightmargin=0pt, innerleftmargin=2em]
		 %	\includegraphics[width=0.15\abstractwidth, valign=m]{assets/windows.png}
		 %	\hfill
		 %	\begin{imgminipage}{assets/header-background.eps}[t]{0.7\abstractwidth}
		 %		\Large \textbf{盒装安装媒介说明书}
%
%		 		\normalsize 请仔细阅读说明书并在管理员指导下使用
%		 	\end{imgminipage}
%		 \end{mdframed}

		\includegraphics[width=\abstractwidth]{assets/header-windows.png}


		\begin{mdframed}[hidealllines=true, innerbottommargin=.5em, innertopmargin=0pt]
			\sffamily

			{\centering 警示语 \par}

			无论是否与其它操作系统合用,安装 Microsoft Windows 均存在丢失磁盘上所有内容的风险。(参见【不良反应】)

            本说明中的信息和意见无意构成法律建议,法律建议可通过咨询律师来获得。

			一些硬件制造商未能提供基于 PnP(Plug-n-Play, 即插即用)协议的设备驱动程序模型,或硬件无法归类于 WDF(Windows Driver Framework, 视窗驱动模型)提供的任何一种设备抽象模型。这些硬件无法在 Microsoft Windows 中正常工作。
		\end{mdframed}
	\end{minipage}}
	\setlength{\columnshrink}{\ht\twocolinsert}
	\addtolength{\columnshrink}{\dp\twocolinsert}
	\noindent\usebox{\twocolinsert}


	\begin{tcolorbox}
	\section*{发行版名称}
	\end{tcolorbox}
	\begin{tabularx}{\linewidth}{@{}ll@{}}
		通用名称: & 视窗操作系统 \\
		正式名称: & Microsoft Windows \\
		拉丁学名: & Fenestrae Molles-Parvae \\
	\end{tabularx}

	\medskip


	\begin{tcolorbox}
	\section*{内容}
	\end{tcolorbox}

	通用 x86-64 Microsoft Windows 基本系统,采用定期发行+滚动升级(Windows Update)模式,尽全力提供最新的稳定版软件,由 Microsoft Corporation 对其进行打包、开发与维护。

	% 内核版本:6.4.1

	% 没有版本号。

	\medskip


	\begin{tcolorbox}
	\section*{性质}
	\end{tcolorbox}

	本系统为采用 Windows NT 内核的操作系统,安装后可由 UEFI 或 Legacy 引导方式启动。

	\medskip


	\begin{tcolorbox}
	\section*{适应平台}
	\end{tcolorbox}

	\begin{itemize}
		\item 仅支持使用 x86-64 (AMD64) 架构的计算机、服务器和嵌入式设备。
		\item 本包装盒中的安装媒介适用于何平台以实际为准。
	\end{itemize}


	\begin{tcolorbox}
	\section*{规格}
	\end{tcolorbox}

	1 枚 安装媒介

	\medskip

	\begin{tcolorbox}
	\section*{用法}
	\end{tcolorbox}

	使用 USB 设备引导。

	启动方式根据硬件调整,一般使用 UEFI。

	根据硬件性能和个人需要,调整安装方式:一般而言,使用 WinPE (Windows Preinstallation  Environment, 视窗预安装环境) 安装。

	安装好基本系统并设置完成软件包后,即可酌情选择可选功能。

	\medskip

	\begin{tcolorbox}
	\section*{不良反应}
	\end{tcolorbox}

	Microsoft Windows 使用“微软更新编录”网站提供可以通过公司网络分发的更新列表。您可以将其用作查找 Microsoft 软件更新、驱动程序和修补程序的一站式位置。每个下载项目都将被分配独立且唯一的KB(Knowledge Base,知识库)编号。

	可以在 https://catalog.update.microsoft
 .com处得到详细信息。

	\medskip


	\begin{tcolorbox}
	\section*{注意事项}
	\end{tcolorbox}
	\begin{itemize}[leftmargin=*]

		\item 满足最低的硬件要求

		下表是内存和硬盘的需求。

		{\small\begin{tabularx}{\linewidth}{|X|X|X|X|}
			\hline
			类别 & RAM\newline (最低) & RAM\newline (推荐) & 硬盘 \\
			\hline
			完整安装 & 4GB & 8GB & 64GB \\
			\hline
		\end{tabularx}}

		基于您的需求,也许可以使用低于上表所列的配置完成系统安装。但是多数用户在无视这些建议的情况下会安装失败。

		\item 需要固件的设备

		除了需要设备驱动程序,有些硬件还要在使用之前在固件设置中配置工作模式。

	\end{itemize}


	\begin{tcolorbox}
	\section*{禁忌}
	\end{tcolorbox}

	在操作过程中出现或即将出现下列任何一种情况,请立即停止操作,并准备好系统恢复。

	\begin{itemize}[leftmargin=*]
		\setlength{\itemsep}{0pt}
		\setlength{\parskip}{0pt}
		\setlength{\parsep}{0pt}

		\item 以管理员权限在 C:{\textbackslash}Windows 目录下执行递归删除
		\item 未经确认即删除所有杀毒软件
		\item 未确认设备盘符即执行格式化
		\item 未确认操作即使用命令重定向
		\item 未经确认即执行来自网络的脚本
		\item 长期在散热不良的设备上高负载使用
	\end{itemize}

	% \begin{tcolorbox}
	% \section*{RISC-V 平台安装}
	% \end{tcolorbox}
	%
	% 所有版本的 Microsoft Windows 不用于 RISC-V 平台。
	%
	% \medskip
	%
	% \begin{tcolorbox}
	% \section*{ARM 平台安装}
	% \end{tcolorbox}
	%
    % Microsoft Windows(x86\_64) 不用于 ARM 平台, ARM 平台需要在管理员指导下安装 Microsoft Winddows on arm。
	%
	% \medskip


	% \begin{tcolorbox}
	% \section*{版本迭代}
	% \end{tcolorbox}


	\begin{tcolorbox}
	\section*{系统相互作用}
	\end{tcolorbox}
	\begin{itemize}[leftmargin=*]
		\setlength{\parindent}{0pt}

		\item 与 Linux 的相互作用

		当您有双引导时,若另一个操作系统与 Windows 访问相同的文件系统,有可能会导致问题和数据丢失。在这种情况下,文件系统的真实状态可能与 Windows 认为在“启动”之后的情况不同,并且可能在进一步写入文件系统时导致文件系统损坏。因此,在双引导设置中,为了避免文件系统损坏,有必要在 Windows 中禁用“快速启动”功能。

		在罕见情况中已观察到,在使用 Windows 进行系统更新时,可能会出现重新启动后引导被破坏从而导致 Linux 无法启动的情况。同时,若在安装过程中将引导信息写入与 Windows 所在的物理磁盘 MBR 内,将导致后者无法正常启动。

        \item 与 x86\_64 架构 Mac 设备上的 macOS 的相互作用

        Windows 在此环境中可能需要专用驱动才能充分利用 Mac 硬件功能。

        用户也可通过虚拟化软件(如 Parallels Desktop)在不重启的情况下同时运行两系统。

        Apple Silicon 架构 Mac 设备不支持传统 Boot Camp 安装方式

		\item 与其他 Windows 版本的相互作用

		与早于 Windows Vista 的版本共存时,可能导致早期版本的 Windows 无法正常启动。

	\end{itemize}


	\begin{tcolorbox}
	\section*{贮藏}
	\end{tcolorbox}

	-40℃\textasciitilde +70℃

	妥善贮藏所有安装媒介,勿使非安装人员触及。

	\medskip


	\begin{tcolorbox}
	\section*{包装}
	\end{tcolorbox}

	刻录有 Microsoft Windows 安装映像的、兼容 USB 3.0 协议的 8GiB 闪存驱动器。

	1 枚/盒

	\medskip


	\begin{tcolorbox}
	\section*{有效期}
	\end{tcolorbox}

	尚不明确。

	\medskip


	\begin{tcolorbox}
	\section*{执行标准}
	\end{tcolorbox}
	\begin{tabularx}{\linewidth}{@{}ll@{}}

	Microsoft 软件许可条款

	\end{tabularx}

	\medskip


	\begin{tcolorbox}
	\section*{批准文号}
	\end{tcolorbox}

	说明书使用 CC-BY-SA 3.0 协议授权。

	\medskip


% 	\begin{tcolorbox}
% 	\section*{生产单位}
% 	\end{tcolorbox}
%
% 	Debian 计划
%
% 	\medskip


	\begin{tcolorbox}
	\section*{说明书}
	\end{tcolorbox}
	\begin{tabularx}{\linewidth}{@{}ll@{}}
		\multirow{2}{*}{}{编审:} & @YukariChiba \\
		~ & @moesoha \\
        ~ & @Tibrella \\
        ~ & @fhh200000 \\
        ~ & @wirano \\
		图形: & @YJBeetle\\
        ~ & @Isoheptane \\
		GitHub: & fhh200000/various-os-media \\
        & -boxes \\
	\end{tabularx}

	\medskip


	\vfill
	\begin{flushright}
		Windows \linebreak 11
		\linebreak
		\newline
		\begin{minipage}{0,5\textwidth}
			\centering
			$\vcenter{\hbox{\includegraphics[height=10mm]{assets/microsoft.png}}}$
		\end{minipage}
	\end{flushright}

\end{multicols*}
\end{document}


